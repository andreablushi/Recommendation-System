\section{Introduction}
In modern business processes, the ability to predict outcomes and provide recommendations based on historical data is crucial for improving efficiency and supporting informed decision-making.
This project investigates the use of process monitoring and machine learning techniques to analyze a real-life production log and to predict whether a process instance will lead to a positive outcome (fast trace: $\text{cycle-time} < \text{avg-cycle-time}$) or a negative outcome (slow trace).
When a negative outcome is predicted, the approach generates recommendations that suggest modifications in the sequence of activities in order to steer the process towards a positive result.
The effectiveness of the recommendation system is then evaluated by comparing the predicted outcomes with the actual outcomes in a held-out test dataset.